\documentclass[
10pt, % Main document font size
a4paper, % Paper type, use 'letterpaper' for US Letter paper
oneside, % One page layout (no page indentation)
BCOR5mm, % Binding correction
]{scrartcl}
\usepackage{amsmath}
\title{Theoretical point of view on the nanowire Josephson junction}
\date{\today}
\begin{document}
\maketitle

\section{Analytics}

We set up to model the system of a nanowire Josephson junction. We model the nanowire as a 3D cylinder of spin-orbit and Zeeman-coupled material:
\begin{align}
H = &\left(\frac{\mathbf{k}^2}{2m}-\mu\right)\sigma_0 \tau_z + \alpha (k_x \sigma_y - k_y \sigma_x)\tau_z \nonumber \\ &+ g \mu_B \mathbf{B}\cdot\boldsymbol{\sigma} + \mathrm{Re}\,\Delta(x, y, z) \sigma_0 \tau_x + \mathrm{Im}\,\Delta(x, y, z) \sigma_0 \tau_y.\label{eq:H}
\end{align}
Here $m$ is the effective mass of carriers in the wire, $\mu$ is the chemical potential, it effectively controls the number of modes in the wire, $\alpha$ is the strength of the Rashba spin-orbit interaction, $\mathbf{B}$ is the strength of magnetic field with $g$ being the $g$-factor. $x$ is the direction along the wire, $y$ is the direction perpendicular to the wire, but in plane of the substrate, and $z$ is perpendicular to the substrate. Due to the electric field generated by the substrate being most likely in $z$ direction, we chose the form of the Rashba term as in \eqref{eq:H}.

Additionally we include orbital magnetic field via Peierls substitution. It works as follows: if the magnetic field is perpendicular to the wire, for example in $y$ direction, then hopping in $z$ direction is modified according to $t_{i,j,k;i,j,k+1} \to t_{i,j,k;i,j,k+1} e^{i \Phi(B, a) x\sigma_0}$ (with $\pm$ signs in electron-hole bases), where $x$ is between the superconductors. The hopping is unmodified on the left superconductor, and on the right it gets multiplied by $e^{i \Phi(B, a) L\sigma_0}$ with L being the distance between the two superconductors. The substitution does not influence $\Delta$ terms in the Hamiltonian.

Before proceeding with the detailed modeling of the Hamiltonian \eqref{eq:H} we estimate the most possible cause of the supercurrent oscillations as a function of the applied parallel magnetic field. 

\begin{itemize}
\item 
Zeeman + disorder. 

This possible origin of the supercurrent oscillations is the combination of the Zeeman field and backscattering inside the Josephson junction\cite{Yok13}. The characteristic scale for the magnetic field of supercurrent oscillations is determined by $\theta_B = \frac{E_Z L}{v_F}$. Here $E_Z$ is the Zeeman energy, $L$ is the length of the nanowire junction, and $v_F$ is the Fermi velocity in the nanowire. Experimental data shows the characteristic scale of magnetic field for the oscillation is $B\approx 0.2$T, which corresponds to $E_Z\approx 0.32$meV, and $L\approx 1\mu$m. To estimate Fermi velocity given the effective mass of the carriers, we assume 5 transverse modes in the nanowire and square crossection of width $W=100$nm. Then the Fermi energy will be $\approx \frac{4\pi^2}{m W^2}$, and Fermi velocity is $v_F \approx \frac{3\pi}{2m W}$. Combining these into the expression for $\theta_B$ and restoring $\hbar$ we get:
\begin{align}
\theta_B \approx \frac{2 E_Z m L W}{3 \pi \hbar^2}\approx 0.04\pi.
\end{align}
Thus the contribution from this effect becomes important only at field $\sim 2$T, long after the oscillations kick in the experiment.

\item Zeeman + SO + disorder

Scenario where SO interaction help create oscillations of supercurrent was discussed in\cite{Yok14}. The characteristic parameter then is $\theta_{SO} = \frac{\alpha k_F L}{\hbar v_F} \propto L/L_{SO}$. This parameter can be non-negligible in the nanowire. However, only paired with $\theta_B\sim \pi/2$ can the system undergo oscillation of supercurrent, see Fig. 3 of \cite{Yok14}. Therefore inclusion of spin-orbit does not move the threshold of oscillations significantly lower.

\item Orbital effect

The crossection of the nanowire is $\pi\times (50\mathrm{nm})^2$, which corresponds to the field of $0.26$T producing a flux quantum penetrating the crossection of the nanowire. Therefore the orbital effect gives the most probable explanation for the observed supercurrent dependence. This does not take into account the effect of magnetic field squeezing due to the superconductor, which can push the effective field scale even lower.

\end{itemize}

\section{Numerics}

To test the above conjecture we have performed numerical simulation of the Hamiltonian \eqref{eq:H} on a lattice. We used the lattice constant $a=8$nm, radius of the wire $R=6$ lattice sites. We have modeled the superconductor as made of the same material as the nanowire, but with a large superconducting gap $\Delta = 20$meV. We have checked that this corresponds to induced gap of $1$meV in the nanowire under the superconductor. 

\begin{thebibliography}{100}
\bibitem{Yok13}
T.~Yokoyama, M.~Eto, and Yu.~V.~Nazarov, J. Phys. Soc. Jpn. \textbf{82}, 054703 (2013).

\bibitem{Yok14}
T.~Yokoyama and Yu.~V.~Nazarov, Euro Phys. Lett. \textbf{108}, 47009 (2014).
\end{thebibliography}

\end{document}